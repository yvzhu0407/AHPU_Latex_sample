%此文档用于引用所需package

\usepackage[%paperwidth=18.4cm, paperheight= 26cm,  % 版面控制宏包,定义规定的版面尺寸
            body={14.5true cm,21true cm}       %论文版芯145mm×210mm(包括页眉及页码则为145mm×230mm)
                               %页码在版芯下边线之下隔行居中放置;
            %headheight=1.0true cm
            ]{geometry}
\usepackage{layouts}				% 打印当前页面格式的宏包
\usepackage[sf]{titlesec}			% 控制标题的宏包
\usepackage{titletoc}				% 控制目录的宏包
\usepackage[perpage,symbol]{footmisc}   % 脚注控制
\usepackage{fancyhdr}				% fancyhdr宏包 页眉和页脚的相关定义
\usepackage{fancyref}        
   
\usepackage{CJK}			% 中文支持宏包
\usepackage{type1cm}		% tex1cm宏包,控制字体的大小
\usepackage{times}			% 使用Times字体的宏包
\usepackage{indentfirst}	% 首行缩进宏包
\usepackage{color}			% 支持彩色